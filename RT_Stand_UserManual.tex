\documentclass[11pt]{article}
\usepackage{tabu}

\begin{document}
\begin{titlepage}
   \vspace*{\stretch{1.0}}
   \begin{center}
      \Large\textbf{Run Tank Stand User Manual}\\
      \large\textit{Robin Liu - Mech 2018}
      
      Dec 2017
   \end{center}
   \vspace*{\stretch{2.0}}
\end{titlepage}

\section{Scope}
This document does not provide the engineering analysis behind the design, but instead just provides some insight into how to use the thing. If you are interested in a (somewhat outdated) analysis of that, search on the drive for my work term report that I wrote on the subject.

This document \textit{does} discuss the water jacket and load cell.

\section{Consumables/Replaceables}
\begin{enumerate}
	\item O Rings are -360 and -366 size. You need at least two of each.
	\item All bolts that touch the vertical columns are 1/4-20, and should be between 1 and 1.5 inches long, inclusive. For your sanity they should be fully threaded. By my count around 50 are required. Accordingly, about 100 washers and 50 nuts are needed.
	\item The bolts connecting the base plate to the legs are 3/8-16. They need to be at least 2 inches long, plus or minus 0.25 (longer is ok just makes it progressively more annoying) The partially threaded 2" bolts with 1" threaded are good for this.
	\item Screws for the feet are \#10-24, and at least 2" long
\end{enumerate}

\section{Full Bill of Materials}
Work in progress
\begin{center}
	\begin{tabu} to \linewidth { |X[l]|X[2,l]|X[l]|X[0.7,r]| }
		\hline
		\textbf{Item Name} & \textbf{Item Description} & \textbf{Suggested Source} & \textbf{Quantity} \\
		\hline
		Columns & 1.5"x1.5"x1/8" Steel Angle, 7.5 feet in length & E3 & 4 \\
		\hline
		Top Plate & 1'x1'x1/8" Steel Plate & E3 & 1 \\
		\hline
		Base Plate & 13"x24"x1/4" Steel Plate & E3 & 1 \\
		\hline
		Legs & 1"x1"x1/8" Square Tube & E3 & 2 \\
		\hline
		
	\end{tabu}
\end{center}


\end{document}
